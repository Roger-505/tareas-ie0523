\section{Resumen}

\hspace{0.35cm} El proyecto desarrollado consiste en un controlador automatizado para la entrada de un estacionamiento, 
implementado mediante Verilog y sintetizado utilizando las librerías RTLIL y \code{cmos\_cells.lib} en Yosys. 
Este sistema controla el acceso de vehículos al estacionamiento, validando la entrada 
correcta y gestionando el estado de la puerta según los estímulos de entrada.

\hspace{0.35cm} Se realizaron pruebas exhaustivas para validar el funcionamiento del controlador en diferentes configuraciones 
de síntesis. Las simulaciones con la librería \code{cmos\_cells.lib} mostraron un comportamiento correcto del módulo 
en todas las pruebas, mientras que la síntesis con \code{RTLIL} presentó problemas con la representación de ciertas 
variables en forma de onda, indicando condiciones no importa en algunas de las señales. Fuera de esto, el módulo RTLIL funcionaba correctamente.
Además de esto, se simuló el comportamiento del controlador de estacionamiento cuando las celdas de \code{cmos\_cells.lib} poseían retardos. 
A pesar de esto, se validó el correcto funcionamiento del módulo sintetizado ante estas condiciones. 

\hspace{0.35cm} La síntesis con \code{cmos\_cells.lib} es más fiable para 
este tipo de diseño, ya que no presentó problemas con condiciones no importa. 
Se recomienda visualizar los mensajes desplegados por Yosys a la hora de realizar la síntesis, ya que estos ayudan a debuggear código no sintetizable. 
