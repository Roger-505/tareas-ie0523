\section{Descripción Arquitectónica}
A continuación, se describe el diseño que se realizó para implementar el controlador automatizado para la entrada de un estacionamiento de forma conductual, en base a las especificaciones indicadas en el enunciado de esta tarea.
El módulo que se realizó para lograr esto se denomina \code{behavioral\_parkingController}.
\subsection{Diagrama de bloques}
Se generó el siguiente diagrama de bloques para el módulo \code{behavioral\_parkingController}. 
Este diagrama incluye las señales de entrada, salida, y registros internos del diseño. 

\begin{figure}[!h]
\centering
\begin{tikzpicture}
    \matrix (m) [
        matrix of nodes,
        nodes in empty cells,
        nodes = {
            text height=2ex, text depth=0.5ex,
            inner sep=1mm, outer sep=0mm,
        },
        row sep = 0mm, column sep = 7mm,
        column 1/.style = {nodes={align=right, anchor=south east}},
        column 2/.style = {nodes={align=center, anchor=south, text width=34ex}}, % change width of block here
        column 3/.style = {nodes={align=left, anchor=south west}},
    ] {
    clk              &                                                  &                   \\
    rst              &                                                  &   open\_gate      \\
    vehicle\_arrival &                                                  &   close\_gate     \\
    vehicle\_left    &      {\vspace{-1.75cm}\\
                            correct\_code $\coloneqq$ 0x2468\\
                            \textbf{behavioral\_parkingController}\\  
                            state[5:0]\\
                            next\_state[5:0]\\
                            attempt[1:0]\\
                            next\_attempt[1:0]}                         &   blocked\_gate   \\
    code\_ack        &                                                  &   wrong\_ping     \\
    gate\_ack        &                                                  &                   \\
    code[15:0]       &                                                  &                   \\
    };
    \scoped[on background layer]
        \node (enc)  [draw, rounded corners, semithick, fill=gray!10,
                      inner sep = 0mm, outer sep= 0mm,
                      fit=(m-1-2) (m-7-2)] {};
    \foreach    \i in {1,...,7}
        \draw[-{Triangle[angle=60:2pt 4]}]    (m-\i-1) -- (m-\i-2);
    \foreach    \i in {2,...,5}
        \draw[-{Triangle[angle=60:2pt 4]}]    (m-\i-2) -- (m-\i-3);
\end{tikzpicture}
\caption{Diagrama de bloques del módulo \code{behavioral\_parkingController}}
\label{fig1}
\end{figure}


\newpage

\subsection{Descripción señales}
En la siguiente tabla, se describen cada una de las señales mostradas en la \hyperref[fig1]{Figura 1}.
Para efectos de esta tarea, todas las señales son activas en alto.

\input{figs/señales}

\newpage

\subsection{Descripción funcional}
Se generó el siguiente diagrama de estados que describe el funcionamiento de la máquina de estados del módulo \code{behavioral\_parkingController}.

\tikzset{
node distance=7.25cm,
every state/.style={thick, fill=gray!10},
auto, semithick,
initial text=$ $,
}

\begin{figure}[!h]
\centering
\begin{tikzpicture}
    \node[state, initial] (novehicle) {NO\_VEHICLE};
    \node[state, right of=novehicle](waitpin) {WAIT\_FOR\_PIN};
    \node[state, right of=waitpin](incorrectpin) {INCORRECT\_PIN};
    \node[state, below of=incorrectpin, align=center](blocked) {BLOCKED \\ \code{blocked\_gate = 1}\\ $\xor \code{wrong\_ping = 1}$};
    \node[state, left of=blocked, align=center](correctpin) {CORRECT\_PIN \\ \code{open\_gate = 1}};
    \node[state, below of=novehicle, align=center](closegate) {CLOSE\_GATE \\ \code{close\_gate = 1}};

    \draw[arrows ={-Latex}]   
        (novehicle)     edge[loop above]    node{$\overline{\code{vehicle\_arrival}}$} (novehicle)
        (novehicle)     edge[above]         node{\code{vehicle\_arrival}} (waitpin)
        (waitpin)       edge[loop above]    node{$\overline{\code{code\_ack}}$} (waitping)
        (waitpin)       edge[bend left]     node[align=center]{\code{code\_ack} $\wedge$\\ $\code{code} \neq \code{correct\_code}$} (incorrectpin)
        (incorrectpin)  edge[left]          node{$\code{attempt\_counter}=3$} (blocked)
        (incorrectpin)  edge[bend left]     node{$\code{attempt\_counter} < 3$} (waitpin)
        (blocked)       edge[loop below]    node{$ $} (blocked)
        (correctpin)    edge[above]         node[align=center]{$\code{vehicle\_left}\wedge$ \\ \code{vehicle\_arrival}} (blocked)
        (correctpin)    edge[loop below]    node{$\overline{\code{vehicle\_left}}$} (correctpin)
        (waitpin)       edge[left]          node[align=center]{\code{code\_ack} $\wedge$\\ $\code{code} = \code{correct\_code}$} (correctpin)
        (correctpin)    edge[above]         node{\code{vehicle\_left}} (closegate)
        (closegate)     edge[left]          node{\code{gate\_ack}} (novehicle);
        

\end{tikzpicture}
\caption{Diagrama de estados del módulo \code{behavioral\_parkingController}}
\label{fig2}
\end{figure}



Este diagrama hace precisamente lo indicado en la especificación del controlador en el enunciado de la tarea. 
Para la codificación de estados en Verilog, se escogió la codificación one-hot. 
Esta permite que solo cambie un bit a la vez, lo cual disminuye las probabilidades de que se llegue a un estado con codificación desconocida al implementarse en hardware. 
Hay cuatro funciones adicionales que no se indican en este diagrama. Estas son:
\begin{itemize}
    \item En caso de que la señal \code{rst} se ponga en alto, se reinicia el módulo, realizando las acciones especificadas en la descripción de esta señal en la \hyperref[t1]{Tabla 1}.
    \item En caso de que el controlador llegue a un estado cuya codificación sea desconocida para el controlador, se ejecutan las mismas acciones indicadas para el punto anterior.
    \item En el estado \code{BLOCKED}, se activa una de dos alarmas, dependiendo de cual fue el estado anterior. Si se ingresó 3 veces consecutivas un pin incorrecto, se activa la alarma \code{wrong\_ping}. Si un vehículo intentó entrar al estacionamiento mientras otro se encontraba entrando, se activa la alarma \code{blocked\_gate}.
    \item El valor por defecto de \code{correct\_code} es 0x2468. Para efectos de las pruebas realizadas, a este parámetro se le asignó el valor 0x5990. Este valor corresponde a los últimos 4 dígitos de mi carnet, representando en BCD.
\end{itemize}
