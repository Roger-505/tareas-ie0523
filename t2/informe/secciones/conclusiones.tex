\section{Conclusiones y recomendaciones}
\subsection{Conclusiones}
\begin{itemize}
    \item Se logró construir una descripción conductual en el lenguaje de descripción de Hardware Verilog de un controlador automatizado para la entrada de un estacionamiento, adecuado a las especificaciones en el enunciado de la tarea.
    \item Se logró verificar el módulo en Verilog generado para el controlador por medio de 4 pruebas, las cuales comprobaron el correcto funcionamiento del módulo. 
\end{itemize}
\subsection{Recomendaciones}
\begin{itemize}
    \item Utilizar archivos \code{*.ttf} (Translate Filter File) para poner etiquetas en las formas de ondas deseadas, y facilitar la lectura de las mismas. 
    \item Utilizar un sistema de control de versiones como \code{git}, ya que permite tener orden a lo largo del flujo de desarrollo del código fuente. 
    \item Utilizar archivos \code{Makefile} modulares para poder hacer cambios rápidos en la construcción de las simulaciones, como agregar nuevos archivos de código fuente o nuevas reglas de compilación.
\end{itemize}
