\section{Instrucciones de utilización de la simulación}
A continuación, se detallan los prerequisitos necesarios para ejecutar las pruebas descritas anteriormente, así como los comandos necesarios para correr la simulación. 

\subsection{Prerequisitos}
\textbf{Nota:} Se aseguró que los comandos que se especificarán a continuación sirvieran únicamente si el usuario corriendo estas simulaciones se encuentre en un sistema operativo Debian o Ubuntu.

Para correr la simulación con las pruebas descritas anteriormente, se requieren las siguientes herramientas de software:
\begin{itemize}
    \item \code{iverilog}: Compilador del lenguaje de descripción de hardware Verilog
    \item \code{make}: Herramienta que automatiza la compilación de programas según las reglas definidas en un \code{Makefile}.
    \item \code{gtkwave}: Visor de formas de onda.
    \item \code{git}: Sistema de control de versiones de código fuente.
\end{itemize}

Puede instalarlos por medio de los siguientes comandos:

\begin{minted}[bgcolor=lightgray]{bash}
sudo apt update && sudo apt upgrade
sudo apt install iverilog make gtkwave git 
\end{minted}

\subsection{Uso}
Clone el repositorio:
\begin{minted}[bgcolor=lightgray]{bash}
git clone https://github.com/Roger-505/tareas-ie0523.git
\end{minted}
Navegue al directorio \code{src} y genere las simulaciones:
\begin{minted}[bgcolor=lightgray]{bash}
cd tareas-ie0523/t2/src    
make clean wave
\end{minted}
Se desplegarán 3 ventanas de \code{gtkwave} una consecutiva a la otra con las formas de onda de las 4 pruebas descritas en el informe de la Tarea \#1.
