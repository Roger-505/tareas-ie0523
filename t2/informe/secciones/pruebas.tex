\section{Plan de pruebas}
El conjunto de pruebas mínimas que se utilizó para validar el diseño conductual consistió de 3 pruebas, cada una con un conjunto de vectores de entrada.
Cada una de estas pruebas es explicada en las secciones, y se marca con un \cmark si el diseño pasó la prueba, o con un \xmark si el diseño la falló.

\subsection{Funcionamiento normal básico \cmark}
En esta prueba, se simuló la llegada de un vehículo a la entrada del estacionamiento. 
A continuación, se describen los vectores de entrada que se generaron para esta prueba:
\begin{itemize}
    \item Ya que esta es la primera prueba que se realiza en la simulación, se reinicia el módulo \code{behavioral\_parkingController}, y todas las señales de entrada se inicializan en bajo, con la excepción de la señal \code{vehicle\_left}, la cual se inicializa en alto. A pesar de esto, note que esta señal sigue siendo activa en alto.
    \item Se pone en alto \code{vehicle\_arrival} para indicar la llegada de un vehículo, y se ingresa la contraseña correcta $\code{code} = \text{0x5990}$, correspondiente a los últimos 4 dígitos de mi carnet representados en BCD. 
    \item Se pone en alto \code{code\_ack} para confirmar el ingreso del código.
    \item Se abre la compuerta, y el vehículo entra al estacionamiento, para que después el controlador vuelva nuevamente al estado \code{NO\_VEHICLE} cuando \code{gate\_ack} se ponga en alto, lo cual confirma que la compuerta ha terminado de cerrarse.
\end{itemize}
\subsection{Ingreso de pin incorrecto menos de 3 veces \cmark}
En esta prueba, se simuló la llegada de un vehículo, ingresa el pin incorrecto dos veces seguidas, y después se retira. 
A continuación, se describen los vectores de entrada que se generaron en esta prueba:
\begin{itemize}
    \item Se pone \code{vehicle\_arrival} en alto para indicar la llegada de un vehículo, y se ingresa una contraseña incorrecta ($\code{code} \neq \text{0x5990}$).
    \item Se pone en alto \code{code\_ack} para confirmar el ingreso del código.
    \item No se debe abrir la compuerta, se incrementa el contador de intentos en 1.
    \item Se realizan nuevamente los dos pasos anteriores, provocando que \code{attempt} = 2. El vehículo se retira, y el controalador vuelve nuevamente al estado inicial \code{NO\_VEHICLE}
\end{itemize}
\subsection{Ingreso de pin incorrecto 3 o más veces \cmark}
En esta prueba, se simuló la llegada de un vehículo, ingresa el pin incorrecto 3 veces seguidas
A continuación, se describen los vectores de entrada que se generaron en esta prueba:
\begin{itemize}
    \item Se pone \code{vehicle\_arrival} en alto para indicar la llegada de un vehículo, y se ingresa una contraseña incorrecta ($\code{code} \neq \text{0x5990}$).
    \item Se pone en alto \code{code\_ack} para confirmar el ingreso del código.
    \item No se debe abrir la compuerta, se incrementa el contador de intentos en 1.
    \item Se realiza dos veces más los dos pasos anteriores, provocando que \code{attempt} = 3. 
    \item El controlador llega al estado \code{BLOCKED}, y se activa la alarma \code{wrong\_ping}. 
\end{itemize}
\subsection{Alarma de bloqueo \cmark}
En esta prueba, se simuló la llegada de un vehículo, ingresa el pin correcto, y mientras este está ingresando al estacionamiento, un segundo vehículo intenta entrar detrás de el primer vehículo. 
A continuación, se describen los vectores de entrada que se generaron en esta prueba:
\begin{itemize}
    \item Se pone en alto \code{vehicle\_arrival} para indicar la llegada de un vehículo, y se ingresa la contraseña correcta $\code{code} = \text{0x5990}$, correspondiente a los últimos 4 dígitos de mi carnet representados en BCD. 
    \item Se pone en alto \code{code\_ack} para confirmar el ingreso del código.
    \item Se abre la compuerta, y el vehículo entra al estacionamiento, poniendo \code{vehicle\_arrival} en bajo, y \code{vehicle\_left} en alto. 
    \item Antes de que \code{gate\_ack} logre ponerse en alto, \code{vehicle\_arrival} se pone en alto.
    \item El controlador llega al estado \code{BLOCKED}, y se activa la alarma \code{blocked\_gate}. 
\end{itemize}
